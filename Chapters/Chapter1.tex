% Chapter 1

\chapter{Introducción general} % Main chapter title

\label{Chapter1} % For referencing the chapter elsewhere, use \ref{Chapter1} 
\label{IntroGeneral}

%----------------------------------------------------------------------------------------

% Define some commands to keep the formatting separated from the content 
\newcommand{\keyword}[1]{\textbf{#1}}
\newcommand{\tabhead}[1]{\textbf{#1}}
\newcommand{\code}[1]{\texttt{#1}}
\newcommand{\file}[1]{\texttt{\bfseries#1}}
\newcommand{\option}[1]{\texttt{\itshape#1}}
\newcommand{\grados}{$^{\circ}$}

%----------------------------------------------------------------------------------------

%\section{Introducción}

%----------------------------------------------------------------------------------------
\section{Deteccion facial}
La vision artificial es un campo cientifico interdisciplinario que se encarga de como los sistemas computacionales pueden obtener un entendimiento de alto nivel de imagenes y videos digitales, para comprender y automatizar tareas como lo haria un sistema de vision humano. Las tareas que ejecuta un sistema de vision artificial son de adquisicion, procesamiento, analisis y entendimiento de imagenes. Un sistema de vision artificial esta compuesto de los siguientes elementos.

% hablar de los campos de estudio %

Uno de los campos de estudio mas importantes de la vision artificial es la deteccion facial. La deteccion facial puede ser considerada como un caso particular de la deteccion de objetos y tiene los objetivos de detectar y localizar todos los rostros humanos contenidos en una imagen digital. En la figura ...

% poner imagen %

Hoy en dia, muchos dispositivos comerciales y profesionales como smartphones, tablets y robots, utilizan la deteccion facial como primer paso para otro tipo de aplicaciones mas complejas, entre las que destacan:
- reconocimiento facial,
- computacion afectiva,
- grabacion de video inteligente

%----------------------------------------------------------------------------------------
\section{Redes neuronales convolucionales}

%----------------------------------------------------------------------------------------
\section{Servicios en la nube}

%----------------------------------------------------------------------------------------
\section{Motivacion}

%----------------------------------------------------------------------------------------
\section{Estado del arte}

%----------------------------------------------------------------------------------------
\section{Objetivos y alcance}

%----------------------------------------------------------------------------------------
\section{Requerimientos}



%----------------------------------------------------------------------------------------